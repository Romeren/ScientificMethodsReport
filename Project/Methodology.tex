Describes which method has been used in the attempt to classify atlantic right whales from images.

\subsection{Preprocessing}

\subsubsection{Clustering}

\subsubsection{Resizing}

\subsection{Classification Algorithm}
Discribes which algorithm was used to classify atlantic right whales. Classification Algorithm have one thing in common which is that they require data input to be on a similar data structure.
The overall data structure for Classification is that in can contain 1..n Observations which each has x amount of data points. For a model, the amount of data points for each observation has to be the same. An example as seen in Table \ref{tab:example data} shows how data could look.
If an observation is missing a value it can be set to NA instead. It still has to be present in the structure.

\begin{table}
  \centering
  \caption{Example data}
  \label{tab:example data}
  \begin{tabularx}{\linewidth}{|l|X|X|X|X|} \hline
    obs. & x1 & x2 & x3 & x4 \\ \hline
    1    & 20 & 74 & 84 & 82 \\ \hline
    2    & 52 & 33 & 4  & 36 \\ \hline
    3    & 78 & 55 & 57 & 3  \\ \hline
    4    & 61 & 68 & 65 & 5  \\ \hline
  \end{tabularx}
\end{table}

\subsubsection{Random Forest}
Random Forest is an algorithm which is based upon the decision trees and wisdom of the crowd.

In order to understand how random forest works. You'd have to understand how decision trees work.
\paragraph{Decision Trees} work on the short plan by taking a decisions as the name indicate. The decision is binary

\subsubsection{Neural Network}