Through the last decades the North Atlantic Right Whale have become a endangered whale spice \cite{NOAA}. The North Atlantic Right Whale have been to the list of animals under the protection of ESA\footnote{The Endangered Spices Act} in 1931. The problem of this convention is that both Japan and the Soviet Union did not the agreement. Consequently, meaning that commercial whaling of the North Atlantic Right Whale to a large extend continued until 1949 where the whale spicy came under the protection of the International Convention for Regulation of Whaling. 
This convention deals with regulation of commercial whiling, but does not prohibit commercial whaling. 
This lead to a over-utilization of whaling in its primary habitat. The result of the commercial over-utilization meant that in 1970 the whale were determined to be in danger of extinction. 
Further, over the next decade, have it been added to both Endangered Species Convention Act, and designated the Marine Mammal Protection Act (MMPA) as Depleted.

In 2015 it have been estimated that there is fewer than 500 individual whales left. Nowadays, fishing of the whale is completely prohibited, but there are still a number of threats for the survival of the North Atlantic Right Whale.
These threats are:
\begin{itemize}
\item Ship collisions 
\item Entanglement of fishing gear
\item Habitat Degradation 
\item Contaminants
\item Climate/Ecosystem changes
\item Disturbance of whale-watching activities
\item Noise from industrial activities
\end{itemize}
To counter these threats and recover the population of the North Atlantic Right Whale a recovery plan was established in 1991 \cite{NOAARecovery}. 
The goal of this plan is to downgrade the status of the whale spicy from endangered to threatened.
To accomplish this goal the recovery plan states seven major recommendations to:
\begin{enumerate}
\item Reduce or eliminate injury or mortality caused by ship collision
\item Reduce or eliminate injury and mortality caused by fisheries and fishing gear
\item Protect habitats essential to the survival and recovery of the species
\item Minimize effects of vessel disturbance
\item Continue international ban on hunting and other directed take
\item Monitor the population size and trends in abundance of the species
\item Maximize efforts to free entangled or stranded right whales and acquire scientific information from dead specimens
\end{enumerate} 
This paper will try to contribute to the major task of monitoring the population size and trends. 



\subsection{Monitoring the population}
The task of monitoring population consist of the following major tasks:
\begin{itemize}
\item \textit{General Monitoring of population.}
\item \textit{Monitoring of habitats.}
\item \textit{Emergency response.}
\end{itemize}

Regulated worldwide but Coordinated nationally, and responded to locally. This requires a lot of knowledge sharing in order the coordinate the effort. The problem here is that the different individual local institutions can make better decisions and reduce there effort by having the shared knowledge of all the individual institutions.
Further, as it is know only a few people has the skill needed to actually identify individual whales.

For this reason, the a Photo-identification database have been created. This database contains a set of images of each individual North Atlantic Right Whale. 
These images can then be used by the local institutions to compare what the whales they have seen with the database, thereby ease the identification of a whale and make sure that each different institute talk about the same whale when sharing knowledge. Beside the images it contains a location/sighting history and a log.
The problem that still remains is the accurate identification of a whale. Even though it is possible to brows through the database and compare the observed with the entries in the database unique identification is both time consuming and inaccurate from the non experts.
Even if the identification have to go through the experts, it will so be time consuming that local institutions will have to wait long periods of time for an answer.
Since time is a critical factor in the protection of the whales this would cause big problems.
