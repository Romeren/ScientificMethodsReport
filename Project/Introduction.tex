This paper will try to contribute to the major task of monitoring the population size and trends. 

\subsection{Monitoring the population}
The task of monitoring population consist of the following major tasks:
\begin{itemize}
\item \textit{General Monitoring of population.}
\item \textit{Monitoring of habitats.}
\item \textit{Emergency response.}
\end{itemize}

Regulated worldwide but Coordinated nationally, and responded to locally. This requires a lot of knowledge sharing in order the coordinate the effort. The problem here is that the different individual local institutions can make better decisions and reduce there effort by having the shared knowledge of all the individual institutions.
Further, as it is know only a few people has the skill needed to actually identify individual whales.

For this reason, the a Photo-identification database have been created. This database contains a set of images of each individual North Atlantic Right Whale. 
These images can then be used by the local institutions to compare what the whales they have seen with the database, thereby ease the identification of a whale and make sure that each different institute talk about the same whale when sharing knowledge. Beside the images it contains a location/sighting history and a log.
The problem that still remains is the accurate identification of a whale. Even though it is possible to brows through the database and compare the observed with the entries in the database unique identification is both time consuming and inaccurate from the non experts.
Even if the identification have to go through the experts, it will so be time consuming that local institutions will have to wait long periods of time for an answer.
Since time is a critical factor in the protection of the whales this would cause big problems.
