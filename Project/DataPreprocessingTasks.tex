In the initial dataset there have been identified a number of significant observations which has to be examined before working with the data.
These are:
\begin{itemize}
\item Varying dimensions
\item Large dimensionality
\item Noise in the dataset
\item Positioning \footnote{Positioning have not been covered by this project due to time limitations} of the whale
\item Rotation \footnote{Rotation have not been covered by this project due to time limitations} of the whale
\end{itemize}
These problems have been split into two different preprocessing tasks, Dimensional Reduction and Isolation of the whale.

\paragraph{Dimensional reduction}

As mentioned in Section \ref{sec:descr-of-data}, there are varying dimensionality in the dataset. Beside the varying dimensions, the images are initially too big to process as it is too time consuming to process machine learning algorithms on images with approximating \( 2000 \times 3000\)px in RGB, since the number of dimensions then would be:
\begin{equation}
Dimensions = 3 \times 2000 \times 3000 = 18.000.000
\end{equation} 
It is not given that more dimensions will provide a better result.
Therefore, one preprocessing task consists of downscaling images in the dataset.
The downscale has been used in this project to resize the images into \(45 \times 30\)px, resulting in images of only 4050 dimensions.
Further dimensional reduction can be archieved through gray scale instead of RGB, or simply use binary (\emph{black/white}) pixels generated given a threshold.

\paragraph{Isolating whale}

In order to get more information about the whale it have been decided to develop a system to crop the images to the whale, thereby allowing more information about the actual whale being present in the downscaled images.

\subparagraph{General Principle}
The general principle behind the cropping is an assumption that there is a true difference between what is whale and what is water. We assume that water pixels, even though there are differences, look more alike than whale and water pixels combined.
Because of this it is assumed that a clustering algorithm could separate water from whale.
In order to lower error rate of this separation there are a number of major parameters:
\begin{itemize}
\item \textit{Smoothing.} By smoothing the image before clustering the pixels, the individual values for the different pixels are equalized based on the neighboring pixels. By doing so, water pixels will look more alike as will individual whale pixels.
\item \textit{Scale.} This will presumably have the same effect as smoothing of the image. But will also increase runtime since it will reduce dimensionality.  
\item \textit{Number of clusters.} In order to separate water from whale only two clusters are needed. But this might have some problems if there are more than just water and whale on the image. For instance, water splash pixels distinguish themselves from both water and whale, by having only two clusters there would be no control of which cluster water splashes will end up in. They might in fact get one of the clusters for themselves, leaving water and whale in one cluster.
\item \textit{Distance algorithm.} Chosen algorithm used to determine the distance between pixels might affect how well the separation is done. As mentioned in Section \ref{sec:litterature}, is it suggested to use euclidean distance
\item \textit{Features.} Using RGB or gray scale values might not be the best way of separating the data. 
Another way could be to use redness. Since whale pixels have higher R value than water pixels this would separate water from whale better.
This feature could be defined as each individual pixels R-value minus the average Red-value\footnote{from the RGB values} for all pixels in a image, causing nearly all water pixels feature value to be close to zero and all other values to be different. 
\end{itemize}
