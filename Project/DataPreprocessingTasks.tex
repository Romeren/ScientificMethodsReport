In the initial dataset there have been identified a number of significant observations which has to be looked at before working with the data.
These are:
\begin{itemize}
\item Varying dimensions
\item Large dimensions
\item Noise in the dataset
\item Positioning \footnote{Positioning have not been covered by this project due to time limitations} of the whale
\item Rotation \footnote{Rotation have not been covered by this project due to time limitations} of the whale
\end{itemize}
These problems have been split into two different preprocessing tasks, Dimensional Reduction and Isolation of the whale.

\paragraph{Dimensional reduction}

As mentioned in Section \ref{sec:descr-of-data}, there are varying dimensionality in the dataset. Beside the varying dimensions, the images are too big to process as it is too time consuming to process machine learning algorithms on images with approximating \( 2000x3000\)px in RGB, since the number of dimensions then would be:
\begin{equation}
Dimensions = 3 \times 2000 \times 3000=18.000.000
\end{equation} 
Further is it not given that more dimensions will provide a better result.
Therefore, one preprocessing task is to downscale the images.
The downscale used in this project will be to \(45x30px\), giving the vectors 4050 dimensions.
Further dimensional reduction is to use gray scale instead of RGB, or simply use binary (\emph{black/white}) generated with a threshold.

\paragraph{Isolating whale}

In order to get more information about the whale it have been decided to develop a system to crop the images to the whale, thereby allowing more information about the actual whale being present in the downscaled images.

\subparagraph{General Principle}
The general principle behind the cropping is an assumption that there is a true difference between what is whale and what is water. Further, do we assume that water pixels, even though there are differences, look more alike than whale pixels and water pixels.
Because of this it is assumed that a clustering algorithm could separate water from whale.
In order to lower error rate of this separation there are a number of major parameters:
\begin{itemize}
\item \textit{Smoothing.} By smoothing the image before clustering the pixels, the individual values for the different pixels are equalized based on the neighboring pixels. By doing so, water pixels will look more alike as will individual whale pixels.
\item \textit{Scale.} This will presumably have the same effect as smoothing of the image. But will also increase performance since it will reduce the number of dimensions.  
\item \textit{Number of clusters.} In order to separate water from whale only two clusters are needed. But this might have some problems if there are more than just water and whale on the image. For instance, does water splash pixels distinguish themselves from both water and whale and by having only two clusters there would be no control of which cluster these will end up in. They might in fact get one of the clusters for themselves, leaving water and whale in one cluster.
\item \textit{Distance algorithm.} Which algorithm used to determine the distance between pixels might affect how well the separation is done. As mentioned in Section \ref{sec:litterature}, is it suggested to use an euclidean distance
\item \textit{Features.} Using the RGB or gray scale values might not be the best way of separating the data. Another way could be to use redness (how red a pixel is compared to green and blue) since the whale is more red than the water. Further could a value for the distance from average pixel value, for each pixel be used. This might be performing well since most of the image is water therefore the feature would approximately describe one pixels distance from what is water causing all water pixels to be approximately zero and other pixels to be not zero.
\end{itemize}
